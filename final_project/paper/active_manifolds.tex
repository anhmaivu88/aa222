\documentclass[]{aiaa-tc}% insert '[draft]' option to show overfull boxes

% Zach del Rosario's LaTeX macros, May 2016
% Inspired by Paul Constantine's macros

% Image Macro: \img{filename}{caption}
\newcommand{\img}[2]{
	\begin{figure}[H]
	\centering
	\includegraphics[width=0.6\textwidth]{../images/#1}   % first argument is the file
	\caption{#2}                  % second argument is caption
	\label{fig:#1}                % generate label from first argument
	\end{figure} }

% Table Macro: \tab{filename}{caption}
\newcommand{\tab}[2]{
	\begin{table}[H]
	\centering
	\input{../tables/#1.tex} 	% first argument is filename
	\caption{#2} 				% second argument is caption
	\label{tab:#1} 				% generatae label from filename
	\end{table}
}

% Vector symbol macros
\newcommand{\vsym}[1]{\boldsymbol{#1}}

\newcommand{\va}{\boldsymbol{a}}
\newcommand{\vb}{\boldsymbol{b}}
\newcommand{\vc}{\boldsymbol{c}}
\newcommand{\vd}{\boldsymbol{d}}
\newcommand{\ve}{\boldsymbol{e}}
\newcommand{\vf}{\boldsymbol{f}}
\newcommand{\vg}{\boldsymbol{g}}
\newcommand{\vh}{\boldsymbol{h}}
\newcommand{\vi}{\boldsymbol{i}}
\newcommand{\vj}{\boldsymbol{j}}
\newcommand{\vk}{\boldsymbol{k}}
\newcommand{\vl}{\boldsymbol{l}}
\newcommand{\vm}{\boldsymbol{m}}
\newcommand{\vn}{\boldsymbol{n}}
\newcommand{\vo}{\boldsymbol{o}}
\newcommand{\vp}{\boldsymbol{p}}
\newcommand{\vq}{\boldsymbol{q}}
\newcommand{\vr}{\boldsymbol{r}}
\newcommand{\vs}{\boldsymbol{s}}
\newcommand{\vt}{\boldsymbol{t}}
\newcommand{\vu}{\boldsymbol{u}}
\newcommand{\vv}{\boldsymbol{v}}
\newcommand{\vw}{\boldsymbol{w}}
\newcommand{\vx}{\boldsymbol{x}}
\newcommand{\vy}{\boldsymbol{y}}
\newcommand{\vz}{\boldsymbol{z}}

% Vector symbol + tilde macros
\newcommand{\vta}{\tilde{\boldsymbol{a}}}
\newcommand{\vtb}{\tilde{\boldsymbol{b}}}
\newcommand{\vtc}{\tilde{\boldsymbol{c}}}
\newcommand{\vtd}{\tilde{\boldsymbol{d}}}
\newcommand{\vte}{\tilde{\boldsymbol{e}}}
\newcommand{\vtf}{\tilde{\boldsymbol{f}}}
\newcommand{\vtg}{\tilde{\boldsymbol{g}}}
\newcommand{\vth}{\tilde{\boldsymbol{h}}}
\newcommand{\vti}{\tilde{\boldsymbol{i}}}
\newcommand{\vtj}{\tilde{\boldsymbol{j}}}
\newcommand{\vtk}{\tilde{\boldsymbol{k}}}
\newcommand{\vtl}{\tilde{\boldsymbol{l}}}
\newcommand{\vtm}{\tilde{\boldsymbol{m}}}
\newcommand{\vtn}{\tilde{\boldsymbol{n}}}
\newcommand{\vto}{\tilde{\boldsymbol{o}}}
\newcommand{\vtp}{\tilde{\boldsymbol{p}}}
\newcommand{\vtq}{\tilde{\boldsymbol{q}}}
\newcommand{\vtr}{\tilde{\boldsymbol{r}}}
\newcommand{\vts}{\tilde{\boldsymbol{s}}}
\newcommand{\vtt}{\tilde{\boldsymbol{t}}}
\newcommand{\vtu}{\tilde{\boldsymbol{u}}}
\newcommand{\vtv}{\tilde{\boldsymbol{v}}}
\newcommand{\vtw}{\tilde{\boldsymbol{w}}}
\newcommand{\vtx}{\tilde{\boldsymbol{x}}}
\newcommand{\vty}{\tilde{\boldsymbol{y}}}
\newcommand{\vtz}{\tilde{\boldsymbol{z}}}

% Matrix symbol
\newcommand{\mA}{\boldsymbol{A}}
\newcommand{\mB}{\boldsymbol{B}}
\newcommand{\mC}{\boldsymbol{C}}
\newcommand{\mD}{\boldsymbol{D}}
\newcommand{\mE}{\boldsymbol{E}}
\newcommand{\mF}{\boldsymbol{F}}
\newcommand{\mG}{\boldsymbol{G}}
\newcommand{\mH}{\boldsymbol{H}}
\newcommand{\mI}{\boldsymbol{I}}
\newcommand{\mJ}{\boldsymbol{J}}
\newcommand{\mK}{\boldsymbol{K}}
\newcommand{\mL}{\boldsymbol{L}}
\newcommand{\mM}{\boldsymbol{M}}
\newcommand{\mN}{\boldsymbol{N}}
\newcommand{\mO}{\boldsymbol{O}}
\newcommand{\mP}{\boldsymbol{P}}
\newcommand{\mQ}{\boldsymbol{Q}}
\newcommand{\mR}{\boldsymbol{R}}
\newcommand{\mS}{\boldsymbol{S}}
\newcommand{\mT}{\boldsymbol{T}}
\newcommand{\mU}{\boldsymbol{U}}
\newcommand{\mV}{\boldsymbol{V}}
\newcommand{\mW}{\boldsymbol{W}}
\newcommand{\mX}{\boldsymbol{X}}
\newcommand{\mY}{\boldsymbol{Y}}
\newcommand{\mZ}{\boldsymbol{Z}}

 \usepackage{varioref}%  smart page, figure, table, and equation referencing
 \usepackage{wrapfig}%   wrap figures/tables in text (i.e., Di Vinci style)
 \usepackage{threeparttable}% tables with footnotes
 \usepackage{dcolumn}%   decimal-aligned tabular math columns
  \newcolumntype{d}{D{.}{.}{-1}}
 \usepackage{nomencl}%   nomenclature generation via makeindex
  \makenomenclature
 \usepackage{subfigure}% subcaptions for subfigures
 \usepackage{subfigmat}% matrices of similar subfigures, aka small mulitples
 \usepackage{fancyvrb}%  extended verbatim environments
  \fvset{fontsize=\footnotesize,xleftmargin=2em}
 \usepackage{lettrine}%  dropped capital letter at beginning of paragraph
 % \usepackage[dvips]{dropping}% alternative dropped capital package
 \usepackage[colorlinks]{hyperref}%  hyperlinks [must be loaded after dropping]

 \title{Pursuing Active Manifolds via Nonlinear Optimization}

 \author{
  Zachary R. del Rosario\thanks{PhD Candidate, Aeronautics and Astronautics, 496 Lomita Mall, Stanford CA, AIAA Student Member.}\\
  {\normalsize\itshape
   Stanford University, Stanford CA, 94305, USA}\\
 }

 % Data used by 'handcarry' option
 % \AIAApapernumber{YEAR-NUMBER}
 % \AIAAconference{Conference Name, Date, and Location}
 % \AIAAcopyright{\AIAAcopyrightD{YEAR}}

 % Define commands to assure consistent treatment throughout document
 \newcommand{\eqnref}[1]{(\ref{#1})}
 \newcommand{\class}[1]{\texttt{#1}}
 \newcommand{\package}[1]{\texttt{#1}}
 \newcommand{\file}[1]{\texttt{#1}}
 \newcommand{\BibTeX}{\textsc{Bib}\TeX}

\begin{document}

\maketitle

\begin{abstract}
Abstract
\end{abstract}

\printnomenclature % creates nomenclature section produced by MakeIndex

%-------------------------------------------------
\section{Introduction}
%-------------------------------------------------

\lettrine[nindent=0pt]{O}{ne} of the most challenging difficulties facing high-fidelity modeling is the treatment of high-dimensional parameter spaces: the Curse of Dimensionality. Consider a parameter study on some quantity of interest (QoI) $f$ in a space of dimension $m$; a simple heuristic is to use $10$ points per dimension, in order to well represent the parameter space. Then the total number of sample points is $10^m$. If a computer code implementing our model executes in a fixed time of $1$ second, then our parameter study execution time scales exponentially. Figure \ref{fig:curse_of_dimensionality} depicts the aforementioned scenario.

\begin{wrapfigure}{R}{0.5\linewidth}
 \includegraphics{../images/curse_of_dimensionality}
 \caption{Execution time scales exponentially with the dimension of parameter space.}
 \label{fig:curse_of_dimensionality}
\end{wrapfigure}

The only reasonable strategy to mitigate this challenge is to perform \emph{dimension reduction}, that is, to reduce $m$. One scheme for dimension reduction of this sort is to seek \emph{Active Subspaces} -- linear subspaces in parameter space along which the majority of variation in our QoI is captured. \cite{constantine2015} The Active Subspaces approach gives a `perfect' dimension reduction in the case that our QoI is a Ridge Function; that is, for $\vx\in\mathbb{R}^m$ and $\mA\in\mathbb{R}^{m\times k}$ with $k<m$, we have $f(\vx)=g(\mA^T\vx)$. Note that a Ridge Function is constant along directions which are orthogonal to $\mA$, that is

\begin{equation}
\mW^T\nabla f = 0,\text{ for } \mW^TA = 0. \label{eq:ridge_property}
\end{equation}

\begin{wrapfigure}{R}{1.0\linewidth}
 \includegraphics{../images/surface_plot}
 \caption{Active and Inactive directions for example function. Note that on average, the function changes more along the active directions than the inactive directions; in this case, the change in the inactive direction is exactly zero.}
 \label{fig:as_example}
\end{wrapfigure}

One example of a Ridge Function is $f(\vx) = \frac{1}{2}(.7x_1+.3x_2)^2$. In this case, the Active Subspace approach discovers the Active and Inactive directions, depicted in Figure \ref{fig:as_example}. Note that this gives us a `perfect' dimension reduction, as we can completely neglect changes along the direction $[-.3,.7]^T$. Such Ridge Functions may seem like a contrivance, but they are actually quite common in multivariate Fourier Transforms\cite{pinkus2015} and physical laws in general.\cite{Constantine2016} Nevertheless, Ridge Functions are not the only functional form that arises in practice; while approximate Active Subspaces are useful, it is easy to construct functions which do not admit this sort of low-dimensional structure.

A natural generalization of Active Subspaces is to seek \emph{Active Manifolds}; that is curved subsets of parameter space which capture the variability of a function. Such low-dimensional structures should recover linear subspaces in the case that they exist (i.e. a Ridge Function), and more general spaces when they do not. Note that if a function is differentiable, we can always move along the gradient to capture the full variability of a function -- unfortunately this requires perfect knowledge of the function, which brings us back to the Curse of Dimensionality. In practice we must use a limited number of gradient samples (assumed to be available, say through an adjoint solution\cite{Jameson1988} or automatic differentiation\cite{Rall1981}) to numerically approximate such a manifold.

%-------------------------------------------------
\section{Seeking Active Manifolds}
%-------------------------------------------------
The strategy we will adopt in this work is to generalize the properties of a Ridge Function: By allowing Equation \ref{eq:ridge_property} to vary in space, we arrive at

\begin{equation}
  \label{eq:inactive_manifold}
  \mW(\vx)^T \nabla f(\vx) = 0
\end{equation}%
\nomenclature{$\vx$}{Input vector}%
\nomenclature{$f(\vx)$}{Scalar quantity of interest}%
\nomenclature{$\nabla f(\vx)$}{Gradient of QoI}%
\nomenclature{$\mW(\vx)$}{Manifold vector}%

%------------------------
\subsection{Numerical Approximation}

%-------------------------------------------------
\section{Solver Performance}
%-------------------------------------------------

%-------------------------------------------------
\section{Conclusion}
%-------------------------------------------------

This had been a brief example of some of the more advanced options
available for \LaTeX.
Please see the documentation for each package for extended discussion or
usage.

% produces the bibliography section when processed by BibTeX
\bibliography{bibtex_database}
\bibliographystyle{aiaa}

\end{document}

% - Release $Name:  $ -
